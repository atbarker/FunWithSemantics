\section{Background}

\subsection{Domain Specific Languages}

\subsection{Cryptol}

\subsection{Implementing Cryptographic Algorithms}

\subsection{RC4}

Rivest Cipher 4 (RC4), also known as ARC4, is a relatively common and simple stream cipher that we have 
used for evaluating different languages in terms of their ability to securely and reliably implement 
cryptographic algorithms. In a stream cipher a stream of pseudorandom bits are generated and then combined 
with the plaintext via an operation such as a bitwise exclusive-or (XOR). The cipher consists of two main 
components, a key scheduling algorithm (KSA) and a pseudo-random generator algorithm (PRGA). The key scheduling algorithm 
takes in a supplied key to generate a permutation of all $256$ possible bytes. This is used a seed for the 
PRGA which iterates over the permutation of bytes generating a pseudo-random key stream equal in length to 
the plaintext. This key stream is then combined with the plaintext using a bitwise XOR. The same operation is 
then repeated in order to retrieve the plaintext.
