\section{Background}
There exists minimal research into the area of Domain Specific Languages for Encryption algorithm development and verification. With that in mind, we will discuss what does currently exist in regards to previous research on the topic.

\subsection{Domain Specific Languages}
 A Domain Specific Language is a programming language specifically designed to facilitate purpose driven coding. DSL's can be divided into markup, modeling, and programming. Our exploration of previous work will only discuss Programming Domain Specific Languages. 

According to Mernik \cite{Mernik}, a Domain Specific Language should offer the following advantages over a generic language: DSL's should have notation based on, or similar to, an existing and established syntax, inclusion of abstraction of the intended domain, and the ability for automatic transformation. Specific to Encryption DSL's, the language design should include the notation elements that support Number theory and Advanced Algebraic concepts. Furthermore, the design would need to incorporate the analysis, verification, parallelization, and optimization necessary to support implementation for modern hardware and software systems. Finally, any DSL design for encryption must allow for specific crytptographic data structures that facilitate operations such as multi precision numbers, univariate polynomials, and compositions of polynomials \cite{Agosta}.

In keeping with these specifications, Agosta and Pelosi \cite{Agosta} attempt to create an  Encryption DSL library based on Python. The library supports both unlimited precision and fixed precision data types by allowing each data type to specified by a size extension. Furthermore, in their library it is possible to specify a user-defined data type with a traditional array construct, or for encryption specific data types, modular arithmetic and polynomial data types can called with built in commands. The framework further supports scalar and vector operators, and maintains python style function declaration to better cope with lower level specifications.

Nielsen and Schwartzbach \cite{Nielson} describe a programming language for secure multiparty communication. While this approach does not focus specifically on a language for the development and testing  of encryption algorithms, it does attempt to bridge the gap between high-level requirements and low level development. This focus provides a framework for development with a distinction between client and server with built-in, provably secure tunneling.  While their proposal is an imperative style DSL, they do not formally address how side effects created by primitive operations affect provable security.

A homomorphic encrytption platform is proposed by Bain et al. \cite{Bain}. This platform would support secret sharing execution while discourage leaks by server collusion. The DSL is built on Haskell and C++ and avoids any unusual language restrictions by relying on the Haskell type system. More specifically, the Haskell constructs employed are monads and type classes. Though, it is not clear from the paper whether or not there is an actual formal verification system that is incorporated.  


\subsection{Cryptol}

\subsection{Implementing Cryptographic Algorithms}

\subsection{RC4}

Rivest Cipher 4 (RC4), also known as ARC4, is a relatively common and simple stream cipher that we have 
used for evaluating different languages in terms of their ability to securely and reliably implement 
cryptographic algorithms. In a stream cipher a stream of pseudorandom bits are generated and then combined 
with the plaintext via an operation such as a bitwise exclusive-or (XOR). The cipher consists of two main 
components, a key scheduling algorithm (KSA) and a pseudo-random generator algorithm (PRGA). The key scheduling algorithm 
takes in a supplied key to generate a permutation of all $256$ possible bytes. This is used a seed for the 
PRGA which iterates over the permutation of bytes generating a pseudo-random key stream equal in length to 
the plaintext. This key stream is then combined with the plaintext using a bitwise XOR. The same operation is 
then repeated in order to retrieve the plaintext.
