\section{Background}
There exists minimal research into the area of Domain Specific Languages for Encryption algorithm development and verification. With that in mind, we will discuss what does currently exist in regards to previous research on the topic.

\subsection{Domain Specific Languages}
 A Domain Specific Language is a programming language specifically designed to facilitate purpose driven coding. DSL's can be divided into markup, modeling, and programming. Our exploration of previous work will only discuss Programming Domain Specific Languages. 

According to Mernik \cite{Mernik}, a Domain Specific Language should offer the following advantages over a generic language: DSL's should have notation based on, or similar to, an existing and established syntax, inclusion of abstraction of the intended domain, and the ability for automatic transformation. Specific to Encryption DSL's, the language design should include the notation elements that support Number theory and Advanced Algebraic concepts. Furthermore, the design would need to incorporate the analysis, verification, parallelization, and optimization necessary to support implementation for modern hardware and software systems. Finally, any DSL design for encryption must allow for specific crytptographic data structures that facilitate operations such as multi precision numbers, univariate polynomials, and compositions of polynomials \cite{Agosta}.

In keeping with these specifications, Agosta and Pelosi \cite{Agosta} attempt to create an  Encryption DSL library based on Python. The library supports both unlimited precision and fixed precision data types by allowing each data type to specified by a size extension. Furthermore, in their library it is possible to specify a user-defined data type with a traditional array construct, or for encryption specific data types, modular arithmetic and polynomial data types can called with built in commands. The framework further supports scalar and vector operators, and maintains python style function declaration to better cope with lower level specifications.

Nielsen and Schwartzbach \cite{Nielsen} describe a programming language for secure multiparty communication. While this approach does not focus specifically on a language for the development and testing  of encryption algorithms, it does attempt to bridge the gap between high-level requirements and low level development. This focus provides a framework for development with a distinction between client and server with built-in, provably secure tunneling.  While their proposal is an imperative style DSL, they do not formally address how side effects created by primitive operations affect provable security.

A homomorphic encrytption platform is proposed by Bain et al. \cite{Bain}. This platform would support secret sharing execution while discourage leaks by server collusion. The DSL is built on Haskell and C++ and avoids any unusual language restrictions by relying on the Haskell type system. More specifically, the Haskell constructs employed are monads and type classes. Though, it is not clear from the paper whether or not there is an actual formal verification system that is incorporated.  

With cPLC, Bangerter et al. \cite{Bangerter} propose a compiler and implementation language for two-party cryptographic protocols. The paper focuses mostly on compiler design and implementation leaving only a small discussion regarding the inclusion of the input language. Furthermore, the input language uses branching and loops which are known to cause information leakage. 


\subsection{Cryptol}
Cryptol is a high specification Encryption Domain Specific Language first developed by the National Security Agency and currently maintained by Galois Inc. Cryptol was originally developed for Field Programmable Gate Array based encryption specific hardware \cite{Lewis}. Built on top of Haskell, Cryptol offers the ability to write, secure implementations of algorithms and be immediately verified for correctness. This sets Cryptol apart as implementations are hardware and platform independent for provable correctness. Cryptol uses uniform sequence expression \cite{Lewis}. Bits are handled as an array of booleans, where one is equal to True. Numbers are represented with bit vectors and modulus arithmetic. Polynomials are supported with through traditional mathematical notation. Finally, control flow and loops are avoided as they provide opportunity for information leakage. 

\subsection{Implementing Cryptographic Algorithms}
There are many considerations that must be taken into account while implementing cryptographic algorthims and cipher or more broadly anything related to the security of 
a computing system\cite{CryptoCoding}. One could easily produce multiple papers giving just an overview of the problems programmers face. We limit ourselves to a relatively brief overview.

First is the issue of comparison methods. If a programmer uses a comparison method for two secret pieces of information which possesses a runtime analogous to the length 
of the secret data the program is difficult to a timing attack. In which an adversary derives the length of a secret from the amount of time taken to process the secret. 
Leaking the length of a secret piece of information drastically reduces the search space for a brute force attack against a cipher. A solution for this is to use a 
constant time comparison method which is not common used in many languages. A similar issue exists when branches (loops or recursion) are controlled by secret data. The 
number of iterations can betray the length of the secret.

Compiler or interpreter optimizations must be conciously avoided as they can introduce unintended behavior into the executing program. For instance some compilers will 
optimize out \texttt{memset} operations used to set a piece of memory that once held a secret to all zeros.

Data type selection is crucial in cryptographic applications. All binary data must be represented by unsigned bytes. Not using unsigned data types or some workaround allows 
for buffer overflow and underflow errors to compromise the operational correctness of an implementation. It is also not advisable to use different data types for secret or 
non-secret information as different types will let an adversary identify pieces of data as secret or non-secret.

Lastly many languages either have built in calls to generate random numbers or provide libraries to do so. These random number generators are often not suitable for use 
in cryptographic applications as they do not achieve the required level of entropy. A programmer must be concious of their source of random number and select those that are 
cryptographically secure such as the Linux/Unix kernel's \texttt{/dev/random}.

\subsection{RC4}

Rivest Cipher 4 (RC4), also known as ARC4, is a relatively common and simple stream cipher that we have 
used for evaluating different languages in terms of their ability to securely and reliably implement 
cryptographic algorithms. In a stream cipher a stream of pseudorandom bits are generated and then combined 
with the plaintext via an operation such as a bitwise exclusive-or (XOR). The cipher consists of two main 
components, a key scheduling algorithm (KSA) and a pseudo-random generator algorithm (PRGA). The key scheduling algorithm 
takes in a supplied key to generate a permutation of all $256$ possible bytes. This is used a seed for the 
PRGA which iterates over the permutation of bytes generating a pseudo-random key stream equal in length to 
the plaintext. This key stream is then combined with the plaintext using a bitwise XOR. The same operation is 
then repeated in order to retrieve the plaintext.
