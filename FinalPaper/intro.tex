While the theoretical creation and proving of cryptographic has a strong and rigorous foundation, the implementation of the same algorithms face significant hurdles from challenges presented by generic programming languages. Specifically, widely used languages such as C, Java, LISP, Python pose language specific problems regarding information leakage and lack of performance when implementing modern cryptography \cite{Lewis}. Even more troubling can be the difficulty in simply understanding the written algorithm or writing the algorithm due to a lack of tools \cite{Agosta}. In this project, we discuss previously proposed Encryption Domain Specific Languages and explore the need for an alternative encryption based Domain Specific Language(DSL).  We also study the challenges faced by modern languages by implementing and benchmarking a modern cryptographic algorithm, RC4, in C, Java, Python, and LISP. Furthermore, we also implement RC4 in Cryptol, a modern and maintained Encryption DSL to provide a counterpoint to the generic language implementations with the hypothesis that Cryptol, or another Encryption DSL, can be as performant as the generic languages while improving readability and and providing provable correctness.